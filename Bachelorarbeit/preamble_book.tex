%---------------------------------------Packages---------------------------------------


\usepackage[paper=a4paper]{geometry} %,left=32mm,right=32mm,top=30mm,bottom=30mm
\usepackage[ngerman,english]{babel}
\usepackage{fancyhdr}
\usepackage[utf8]{inputenc}
% \usepackage{lmodern} % Font normal
% \usepackage{tgpagella} % Font palantino
\usepackage[T1]{fontenc}
\usepackage[rgb,dvipsnames]{xcolor}
\usepackage{microtype}
\usepackage[all]{nowidow}
\usepackage{graphicx}
\usepackage{float}
%\usepackage{tikz}
\usepackage{pgfplots}
\usepackage[shortlabels]{enumitem}
\usepackage{amsmath}
\usepackage{amssymb}
\usepackage{amsthm}
\usepackage{siunitx}
\usepackage{booktabs}
\usepackage{mathtools}
\usepackage{wrapfig}
\usepackage[titletoc]{appendix}
\usepackage[nonewpage,quiet]{imakeidx}
\usepackage[linktoc=all]{hyperref}
\usepackage{tocloft}
\usepackage{titlesec}
\usepackage[labelsep=period]{caption}
\usepackage{subcaption}
\usepackage[style=numeric,maxnames=3,sortcites=true,doi=false,isbn=false,eprint=false,url=false]{biblatex}
\usepackage[nameinlink,capitalize]{cleveref}
\usepackage{csquotes}
\usepackage{thmtools} 
\usepackage{xurl}
\usepackage{grffile}
\usepackage{mdframed} %vorher war [framemethod=TikZ]
\usepackage{algorithm}
\usepackage{algpseudocode}
% \usepackage{printlen}
% \usepackage{layouts} # only used to get \textwidth



%---------------------------------------Diverse Befehle---------------------------------------


\addbibresource{Bibliography.bib}
\graphicspath{{./figures/}}
\setlist{nosep} %enumitem: remove vertical spaces around lists


\DeclareCaptionLabelSeparator{bar}{\,|\, }
\captionsetup{font=small,labelfont=bf, labelsep=bar, width=.9\linewidth,belowskip=-7pt}


\numberwithin{equation}{chapter} %gleichungen in kapiteln nummerieren
\allowdisplaybreaks % align umgebungen umbrechen
\setlength{\oddsidemargin}{35pt} 
\setlength{\evensidemargin}{35pt}%macht, dass gerade und ungerade seiten gleiche margins haben
\pgfplotsset{compat=1.18}
\usepgfplotslibrary{external} 
\tikzexternalize

\indexsetup{level=\section,noclearpage,firstpagestyle=fancy}
\makeindex[columns=2, title=Alphabetical Index]


\setcounter{topnumber}{1} %max number of floats at the top of a page
% \makeindex[columns=3, title=Alphabetical Index]


% % Diese Befehle ändern die Farben der Seitennummern
% % im Inhaltsverzeichnis. Möglich durch \usepackage{tocloft}
% \makeatletter
% % \definecolor{bluelinkcolor}{RGB}{35, 60, 138}
% \renewcommand{\cftchappagefont}{\bfseries\color{BlueViolet}}
% \renewcommand{\cftsecpagefont}{\color{BlueViolet}}
% \let\cftsubsecpagefont\cftsecpagefont % \renewcommand{\cftsubsecpagefont}{\HyColor@HyperrefColor{red}\@linkcolor}
% \makeatother




%---------------------------------------Cleverref---------------------------------------



\crefname{chapter}{Chapter}{Chapters}
\crefname{section}{Section}{Sections}
\crefname{subsection}{Subsection}{Subsections}
\crefname{algorithm}{Algorithm}{Algorithms}
\crefname{figure}{Figure}{Figures}
\crefname{table}{Table}{Tables}
\crefname{equation}{Equation}{Equations}
\crefname{problem}{Problem}{Problems}
\crefname{appchap}{Appendix}{Appendices}
\crefname{appsec}{Appendix}{Appendices}



%---------------------------------------PDF Einstellungen---------------------------------------



\hypersetup{
  pdftitle    = {Bachelorarbeit},
  pdfsubject  = {Chance of Further Improvement in Bayesian Optimization},
  pdfauthor   = {Marc Philip Kaebel},
  colorlinks  = false,
  pdfborder   = {0 0 0}
}



%---------------------------------------Clickable Chapter Names leading to TOC------------------



\AtBeginDocument{
  \let\oldchapter\chapter
  \RenewDocumentCommand{\chapter}{s o m}{%
    \clearpage
    \IfBooleanTF{#1}
    {\oldchapter*{\hyperref[toc]{#3}}}% \chapter*[..]{...}
    {\IfValueTF{#2}
      {\oldchapter[#2]{\hyperref[toc]{#3}}}% \chapter[..]{...}
      {\oldchapter[#3]{\hyperref[toc]{#3}}}% \chapter{...}
      \label{chapter-\thechapter}% \label this chapter
    }%
  }
}

%---------------------------------------Fancy Pagestyle Setup-----------------------------------

\setlength{\headheight}{15pt} %fancyhdr

\fancypagestyle{plain}{  % makes the first pages of chapters have page numbers that are at the bottom on the sides
\fancyhf{}
\fancyfoot[LE,RO]{\textbf{\thepage}}
\renewcommand{\headrulewidth}{0pt}
\renewcommand{\footrulewidth}{0pt}}




\pagestyle{fancy}
\renewcommand{\chaptermark}[1]{\markboth{#1}{}}
%\renewcommand{\sectionmark}[1]{\markright{#1}}
\fancyhf{}
%\fancyhead[LE,RO]{\thepage}
\fancyhead[LE]{\nouppercase{\scshape\textbf{\thepage}\,|\, \leftmark}}
\fancyhead[RO]{\nouppercase{\scshape\rightmark\, |\,\textbf{\thepage}}}
\renewcommand{\headrulewidth}{0pt}



%---------------------------------------Chapter and Section Style-------------------------------




\titleformat{\subsection}
{\normalfont\itshape\large}
{\thesection}{0.5em}{\normalfont\normalsize\itshape}
% {\thesection\,|\, }{0.5em}{}




\titleformat{\section}
{\normalfont\bfseries\large}
{\thesection}{0.5em}{\normalfont\normalsize\scshape}
% {\thesection\,|\, }{0.5em}{}



\titleformat{\chapter}[display]
  {\normalfont}{\large\scshape\chaptertitlename\ \thechapter}{0pt}{\LARGE\bfseries}
% \titlespacing*{\chapter}s
%   {0pt}{65pt}{40pt}



%---------------------------------------Thmtools---------------------------------------



\let\proof\relax
\let\endproof\relax



\declaretheoremstyle[
spaceabove=3pt, 
spacebelow=6pt,
headfont=\normalfont\bfseries,
notefont=\mdseries, 
notebraces={(}{)},
bodyfont=\normalfont]{standardthm}

\declaretheoremstyle[
spaceabove=3pt, 
spacebelow=6pt,
headfont=\normalfont\bfseries,%\scshape,
notefont=\normalfont\mdseries, 
notebraces={(}{)},
bodyfont=\normalfont,
postheadspace=1em]{standarddef}

\declaretheoremstyle[
spaceabove=0pt, 
spacebelow=0pt,
headfont=\normalfont\bfseries,
notefont=\mdseries, 
notebraces={(}{)},
bodyfont=\normalfont,
postheadspace=1em]{standardrem}

\declaretheoremstyle[
spaceabove=0pt, 
spacebelow=0pt,
headfont=\normalfont\bfseries,
bodyfont=\normalfont,
postheadspace=1em,
qed=\qedsymbol%$\blacksquare$
]{standardproof}



\declaretheorem[
style=standardthm,
refname={Theorem,Theorems},
Refname={Theorem,Theorems},
numberwithin=chapter]{theorem}

\declaretheorem[
style=standardthm,
sibling=theorem,
refname={Proposition,Propositions},
Refname={Proposition,Propositions}]{proposition}

\declaretheorem[
style=standardthm,
sibling=theorem,
refname={Lemma,Lemmas},
Refname={Lemma,Lemmas}]{lemma}

\declaretheorem[
style=standardthm,
sibling=theorem,
refname={Corollary,Corollaries},
Refname={Corollary,Corollaries}]{corollary}



\declaretheorem[
style=standarddef,
sibling=theorem,
refname={Definition,Definitions},
Refname={Definition,Definitions}]{definition}

\declaretheorem[
style=standarddef,
sibling=theorem,
refname={Example,Examples},
Refname={Example,Examples}]{example}

\declaretheorem[
style=standardrem,
sibling=theorem,
refname={Remark,Remarks},
Refname={Remark,Remarks}]{remark}



\declaretheorem[
style=standardproof,
unnumbered,
refname={Proof,Proofs},
Refname={Proof,Proofs}]{proof}



%------------------------------------------------------------------------------