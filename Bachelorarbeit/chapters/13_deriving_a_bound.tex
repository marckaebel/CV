\chapter{Deriving a Uniform Bound}~\label{chap:bound}
% 
In this chapter we will derive an upper bound on the probability of further improvement of a Kriging process.
For that we will use two different inequalities from \cref{sec:chaining}. The first is Dudley's inequality, the second is Talagrand's inequality. In both cases we use the metric entropy of the space to find a bound on the variance of the process, then use this bound to derive a bound on the probability of further improvement.
We limit ourselves to the case of Gaussian processes with Matérn covariances on underlying $d$-dimensional spaces $T= [0,1]^{d}$.

Suppose our underlying stationary covariance function is $C$. We write $C(x,y) = C(\lvert x-y \rvert)$ for $x,y \in T$.
Further we write the Kriging covariance as
\( C(x,y \mid x_{1}, \dots, x_{n}) \)
for $x,y$ and $x_{1},\dot{,x_{n}} \in T$. 
We know from \cref{eq:kriging_predictor} that the Kriging process has a covariance matrix
\[
    \Sigma_{(x,y) \mid x_{1}, \dots, x_{n}} = \Sigma_{x,y} - \Sigma_{(x,y), x_{1}, \dots, x_{n}}^{T} \Sigma_{(x_{1}, \dots, x_{n}),(x_{1}, \dots, x_{n})}^{-1} \Sigma_{(x,y), x_{1, \dots, x_{n}}}.
\]
Writing out the matrices gives us
\begin{align*}
    \Sigma_{(x,y) \mid x_{1}, \dots, x_{n}} &= 
    \begin{bmatrix}
    C(x,x) & C(x,y)\\
    C(x,y) & C(y,y)
    \end{bmatrix}\\
    &-
    \begin{bmatrix}
    C(x,x_{1}) & \dots & C(x,x_{n}) \\
    C(y,x_{1}) & \dots & C(y,x_{n}) 
    \end{bmatrix}
    \Sigma^{-1}
    \begin{bmatrix}
    C(x,x_{1})  & C(y,x_{1}) \\
    \vdots  & \vdots \\
    C(x,x_{n})  & C(y,x_{n})  
    \end{bmatrix},
\end{align*}
with \( \Sigma^{-1} \coloneqq \Sigma_{(x_{1}, \dots, x_{n}),(x_{1}, \dots, x_{n})}^{-1} = [C(x_{i}),C(x_{j})]_{1 \leq i,j \leq n}^{-1} \).
And hence we can extract formulas for the Kriging covariance functions
\[
    C(x,y \mid x_{1}, \dots, x_{n}) = 
    C(x,y) 
    - 
    \begin{bmatrix}
    C(x,x_{1}) & \dots & C(x,x_{n}) 
    \end{bmatrix}
    \Sigma^{-1}
    \begin{bmatrix}
    C(y,x_{1}) \\
    \vdots \\
    C(y,x_{n}) 
    \end{bmatrix}
\]
and 
\[
    C(x,x \mid x_{1}, \dots, x_{n}) = 
    C(x,x) 
    - 
    \begin{bmatrix}
    C(x,x_{1}) & \dots & C(x,x_{n}) 
    \end{bmatrix}
    \Sigma^{-1}
    \begin{bmatrix}
    C(x,x_{1}) \\
    \vdots \\
    C(x,x_{n}) 
    \end{bmatrix}.
\]
% Later we will write
% $c_{y}=\begin{bmatrix}C(y,x_{1}) & \dots & C(y,x_{n}) \end{bmatrix}^{T}$.
%
%
\section{Preparations}
%
For both the Dudley and Talagrand approaches we will need to bound the covering numbers as well as the variance of the Kriging process $\{ X_{t} \}_{t \in T}$, so a bound on \( \bar{\sigma}^2 \coloneqq \sup_{t \in B} C(t,t)  = \sup_{t \in B} \E[(X_{t} - \E[X_{t}])^2] \) for certain parts of the index set $B \subseteq T$.
We start by bounding the covering numbers.
\begin{lemma}[Hölder continuity of Matérn covariance]
    ~\label{lemma:matern_hoelder}
    Let $C$ denote the Matérn covariance function with parameters \( \nu> d/2 \) and \( m > 0 \).
    For \( 0 < \eta < 2\nu-d \) there exists  a constant \( k >0 \) such that for all \( s,t \in \mathbb{R}^{d} \) 
    \[
        \lvert C(s) -C(t) \rvert \leq k \lvert s-t \rvert^{\eta}.
    \]
\end{lemma}
\begin{proof}
We will follow the proof from \cite[Lemma 4.4]{ernst2020integrability}.
Fix $\eta \in (0,1)$.
Note that for $z,w \in \mathbb{C}$ with $\lvert z-w \rvert \leq 2$ we have
\[
    \lvert z-w \rvert = \lvert z-w \rvert^{1-\eta} \lvert z-w \rvert^{\eta} \leq 2^{1- \eta} \lvert z-w \rvert^{\eta}.
\]
Further note that we have
\[
     \lvert e^{- i \xi x}- e^{-i \xi y} \rvert \leq 2
\]
for all $x,y, \xi \in \mathbb{R}^{d}$ since $e^{-it}$ is on the unit circle for $t \in \mathbb{R}^{d}$.
By an application of the mean value theorem to $f(t)= e^{-i \xi t}$ we have
\[
    \lvert e^{- i \xi x} - e^{- i \xi y} \rvert \leq \lvert \xi (x-y) \rvert.
\]
We can combine these facts and obtain
\begin{align*}
    \lvert C(s)-C(t) \rvert &\leq \frac{1}{(2 \pi)^{d}} \left| \int_{\mathbb{R}^{d}} \frac{e^{-i \xi x} - e^{- i \xi y }}{(\lvert \xi \rvert^2 + m^2)^{\nu}}  \dx[\xi]\right|\\
    &\leq \frac{2^{1-\eta}}{(2\pi)^{d}} \lvert x-y \rvert^{\eta} \int_{R^{d}} \frac{\lvert \xi \rvert^{\eta}}{(\lvert \xi \rvert^2 + m^2)^{\nu}} \dx[\xi].
\end{align*}
To answer the remaining question of when this integral is finite, we bound
\begin{align*}
    \int_{\mathbb{R}^{d}} \frac{\lvert \xi \rvert^{\eta}}{(\lvert \xi \rvert^{2}+m^2)^{\nu}} \dx[\xi] &\leq 
    \int_{\mathbb{R}^{d}} \frac{1}{\lvert \xi \rvert^{2\nu-\eta}} \dx[\xi].
\end{align*}
This integral is finite for $2\nu-\eta > d$, which proves the desired inequality.
\end{proof}
%
%
%
\begin{proposition}[Bounding the covering numbers]
    ~\label{prop:covering_numbers}
    Let $C$ be a Matérn covariance function $C$ with parameters $\nu $ and $m$. Let $\{ G_{t} \}_{t \in T}$ be a Kriging process with measurements $x_{1}, \dots, x_{n}$. We again write $C(x,y) \coloneqq C(\lvert x-y \rvert)$ for the unconditioned process as well as $C(x,y \mid x_{1}, \dots, x_{n}) \coloneqq \Cov(G_{x},G_{y})$ for the Kriging process.
    There exist constants $A$ and $\alpha$, such that the covering numbers of any subset $B \subseteq T$ for $\varepsilon>0$ with regards to the canonical distance are bounded by
    \[
        \mathcal{N}(B,d_{G},\varepsilon) \leq (\frac{A}{\varepsilon})^{\alpha}.
    \] 
\end{proposition}
%
%
\begin{proof}
We adapt a proof from \cite{ernst2020integrability} to the setting of Kriging processes.
We will first show Hölder continuity of the canonical distance of the Kriging process. This will then help us to cover any $\varepsilon$-balls of the canonical distance with certain $\tilde{\varepsilon}$-balls of the euclidean norm. Then we can bound the covering number $\mathcal{N}(B,d_{G},\varepsilon)$ by the covering number $\mathcal{N}(B, \lvert \cdot \rvert, \tilde{\varepsilon})$, which is easy to bound as a function of the diameter of $B$. 
%
%
Let us begin with proving the Hölder continuity of the canonical distance $d_{G}$. 
As a convenience we define $c_{x}=\begin{bmatrix}C(x,x_{1}) & \dots & C(x,x_{n}) \end{bmatrix}^{T}$ for $x \in T$.
We can write the canonical distance for $x,y \in T$ as
\begin{align*}
    d_{G}(x,y)^2 &= \Var(G_{x}-G_{y})\\
    &= \Var(G_{x}) + \Var(G_{y}) - 2 \Cov(G_{x},G_{y})\\
    &= C(x,x \mid x_{1}, \dots, x_{n}) + C(y,y \mid x_{1}, \dots, x_{n}) -2 C(x,y \mid x_{1}, \dots, x_{n})\\
    &= C(0)- c_{x}^{T}\Sigma^{-1}c_{x}+ C(0)- c_{y}^{T} \Sigma^{-1} c_{y} 
    - 2C(\lvert x-y \rvert)+2 c_{x}^{T} \Sigma^{-1} c_{y}\\
    &= 2(C(0)-C(\lvert x-y \rvert)) -(c_{x}-c_{y})^{T}\Sigma^{-1}(c_{x}-c_{y}),
\end{align*}
where we used the fact that $\Sigma^{-1}$ is symmetric to transpose \( c_{x}^{T} \Sigma^{-1} c^{y} = c_{y}^{T} \Sigma^{-1} c^{x}  \).
Now since $\Sigma^{-1}$ is also positive definite, \( (c_{x}-c_{y})^{T}\Sigma^{-1}(c_{x}-c_{y}) \geq 0 \)
and the $d_{G}(x,y)^2$ can be bounded by
\[
    d_{G}(x,y)^2 \leq 2(C(0)-C(\lvert x-y \rvert)).
\]
Applying the Hölder continuity of \cref{lemma:matern_hoelder}, we have
\[
    d_{G}(x,y) \leq \sqrt{2k} \lvert x-y \rvert^{\eta/2}
\]
for $0 < \eta < 2 \nu -d$ with a constant $k$.

We are ready to deal with the covering numbers.
Due to the bound on the canonical distance by the Euclidean norm, we can cover any $\tilde{\varepsilon}$-balls of the Euclidean distance with $\varepsilon$-balls of the canonical distance.
To be precise, we have 
\[
    B_{\lvert \cdot \rvert, \tilde{\varepsilon}}(x) \subseteq B_{d_{G},\varepsilon}(x)
\]
with \( \tilde{\varepsilon} = \left(\frac{\varepsilon^{2}}{2k}\right)^{1/\nu}  \)
Therefore any covering of $B$ with a set of \( B_{\lvert \cdot \rvert, \tilde{\varepsilon}}(x) \) balls gives us a covering of $B$ with the same number of \(  B_{d_{G},\varepsilon}(x) \) balls. Hence 
\[
    \mathcal{N}(B,d_{G},\varepsilon) \leq \mathcal{N}(B,\lvert \cdot \rvert,\tilde{\varepsilon}).
\]
As is shown in \cite[Cor. 4.2.13]{vershynin2020high}, the latter can be bounded by 
\[
    \mathcal{N}(B,\lvert \cdot \rvert,\tilde{\varepsilon}) \leq \left( \frac{2 \diam(B) \sqrt{d}}{\tilde{\varepsilon}} \right)^{d},
\]
where $d$ is the dimension of $T$.
Putting everything together we have
\begin{align*}
    \mathcal{N}(B,d_{G},\varepsilon) &\leq \left( \frac{2 \diam(B) \sqrt{d}}{\tilde{\varepsilon}} \right)^{d}\\
    &= \left( \frac{(2 \diam(B) \sqrt{d})^{\eta} 2k}{\varepsilon^2} \right)^{\frac{d}{\eta}}\\
    &= \left( \frac{ \sqrt{(2 \diam(B) \sqrt{d})^{\eta} 2k }}{\varepsilon} \right)^{\frac{2d}{\eta}}.
\end{align*}

Therefore the bound
\[
    \mathcal{N}(B,d_{G},\varepsilon) \leq (\frac{A}{\varepsilon})^{\alpha}
\]
holds, with $\alpha = \frac{2d}{\eta}$ and $A = \left((2 \diam(B) \sqrt{d})^{\eta} 2k\right)^{1/2}$.
%
%
%
\end{proof}
Having derived a bound on the covering numbers, let us turn to bounding the process variance \( \bar{\sigma}^2 = \sup_{x \in B} C(x,x) \).
\begin{lemma}
    ~\label{lemma:kriging_variance_monotone}
    Let $C(x,y \mid x_{1}, \dots, x_{n})$ with $n \in \mathbb{N}$ and $x,y \in T$ be the covariance function of a Kriging process with an underlying Matérn covariance and measurements $x_{1}, \dots, x_{n} \in T$.
    Then the Kriging variance is monotonely decreasing with added measurements, i.e.
    \[
        C(x,x \mid x_{1}, \dots, x_{n}) \leq C(x,x \mid x_{2}, \dots x_{n}).
    \]
\end{lemma}
\begin{proof}
Before we look at the Kriging covariance functions, we will need a small matrix computation.
We define the covariance matrix
\[ \Sigma = 
\begin{bmatrix}
C(x_{1},x_{1}) & \dots & C(x_{1},x_{n}) \\
\vdots  &  \ddots & \vdots  \\
C(x_{n},x_{1}) & \dots  &  C(x_{n},x_{n})
\end{bmatrix} = 
\begin{bmatrix}
C(x_{1},x_{1}) & \Sigma_{B} \\
\Sigma_{B}^{T}  &  \Sigma_{C}
\end{bmatrix} \] 
with \( \Sigma_{C} \in \mathbb{R}^{n-1,n-1} \) and \( \Sigma_{B} \in \mathbb{R}^{n-1} \).
Since \( C \) is a Matérn covariance function, the matrices $\Sigma$, $\Sigma_{C}$ are positive definite and $C(x_{1},x_{1}) > 0$. Hence $\Sigma$ and $\Sigma_{C}$ are invertible.
We write
\[
    \Sigma^{-1}= \Lambda = \begin{bmatrix}
    \Lambda_{A} & \Lambda_{B} \\
    \Lambda_{B}^{T}  &  \Lambda_{C}
    \end{bmatrix}.
\]
with $\Lambda_{A} \in \mathbb{R}$ and all other dimensions to match.
Computing the blocks of this inverse matrix gives us
\begin{align*}
    \Lambda_{A}&= \left( C(x_{1},x_{1})- \Sigma_{B}\Sigma_{C}^{-1}\Sigma_{B}^{T} \right)^{-1}\\
    \Lambda_{B} &= -\left( C(x_{1},x_{1})- \Sigma_{B}\Sigma_{C}^{-1}\Sigma_{B}^{T} \right)^{-1}\Sigma_{B}\Sigma_{C}^{-1}\\
    \Lambda_{C} &= \Sigma_{C}^{-1}+\Sigma_{C}^{-1}\Sigma_{B}^{T}\left( C(x_{1},x_{1})- \Sigma_{B}\Sigma_{C}^{-1}\Sigma_{B}^{T} \right)^{-1}\Sigma_{B}\Sigma_{C}^{-1}.
\end{align*}
Now according to \cite[Prop. 2.2]{gallier2019schur} applied to $\Sigma$ the Schur complement \( \Sigma/ \Sigma_{C} = C(x_{1},x_{1})- \Sigma_{B} \Sigma_{C}^{-1} \Sigma_{B}^{T}\) is positive definite, therefore
\( C(x_{1},x_{1})- \Sigma_{B} \Sigma_{C}^{-1} \Sigma_{B}^{T} > 0\).
With this established, we can turn our attention to the Kriging covariance functions.
By definition we have
\[
    C(x,x \mid x_{1}, \dots x_{n}) = C(x,x) - c_{x}^{T}\Lambda c_{x}
\]
and 
\[
    C(x,x \mid x_{2}, \dots x_{n}) = C(x,x) - \tilde{c}_{x}^{T}\Sigma_{C}^{-1} \tilde{c}_{x}
\]
with 
\( c_{x} \coloneqq \begin{bmatrix} C(x,x_{1}) &  \dots & C(x,x_{n}) \end{bmatrix}^{T} \) 
and 
\( \tilde{c}_{x} \coloneqq \begin{bmatrix} C(x,x_{2}) &  \dots & C(x,x_{n}) \end{bmatrix}^{T} \).
To show the desired monitonicity, it thus suffices to show 
\[
    c_{x}^{T}\Lambda c_{x} \geq \tilde{c}_{x}^{T} \Sigma_{C}^{-1} \tilde{c}_{x}.
\]
We start by multiplying out the left hand side
\begin{align*}
    c_{x}^{T} \Lambda c_{x} &= C(x,x_{1}) \Lambda_{A} C(x,x_{1}) + C(x,x_{1}) \Lambda_{B} \tilde{c}_{x} + \tilde{c}_{x}^{T} \Lambda_{B}^{T} C(x,x_{1}) + \tilde{c}_{x}^{T} \Lambda_{C} \tilde{c}_{x} \\
    &= C(x,x_{1})^2 \Lambda_{A} + 2 C(x,x_{1}) \Lambda_{B} \tilde{c}_{x} + \tilde{c}_{x}^{T} \Lambda_{C} \tilde{c}_{x}.
\end{align*}
Plugging in from the inverse block matrix $\Lambda$ we have
\begin{align*}
    c_{x}^{T} \Lambda c_{x} &= \frac{1}{C(x_{1},x_{1})-\Sigma_{B}\Sigma_{C}\Sigma_{B}^{T}} ( C(x,x_{1})^2 - 2 C(x,x_{1}) \Sigma_{B} \Sigma_{C}^{-1} \tilde{c}_{x} \\
    & +\tilde{c}_{x} \Sigma_{C}^{-1} \Sigma_{B}^{T} \Sigma_{B}\Sigma_{C}^{-1} \tilde{c}_{x} ) + \tilde{c}_{x}^{T} \Sigma_{C}^{-1} \tilde{c}_{x}\\
    &= \frac{1}{C(x_{1},x_{1})-\Sigma_{B}\Sigma_{C}\Sigma_{B}^{T}} \left( C(x,x_{1})- \Sigma_{B} \Sigma_{C}^{-1} \tilde{c}_{x}  \right)^2 +\tilde{c}_{x}^{T} \Sigma_{C}^{-1} \tilde{c}_{x} \\
    &\geq \tilde{c}_{x}^{T} \Sigma_{C}^{-1} \tilde{c}_{x},
\end{align*}
which proves the monotonicity.
\end{proof}
%
%
Now let us look specifically at the case with only one measurement $x_{1}$. We can write the covariance matrix as
\[
    \Sigma_{(x,y) \mid x_{1}} = \begin{bmatrix}
    C(x,x \mid x_{1}) & C(x,y \mid x_{1}) \\
    C(x,y \mid x_{1}) & C(x,y \mid x_{1}) 
    \end{bmatrix}.
\]
As before,
\begin{align*}
    \Sigma_{(x,y) \mid x_{1}} &= \Sigma_{x,y} - \Sigma_{(x,y), x_{1}}^{T} \Sigma_{x_{1},x_{1}}^{-1} \Sigma_{(x,y), x_{1}}\\
    &= \begin{bmatrix}
    C(x,x) & C(x,y)\\
    C(x,y) & C(y,y)
    \end{bmatrix}
    - \frac{1}{C(x_{1},x_{1})}
    \begin{bmatrix}
    C(x,x_{1})^2            & C(x,x_{1}) C(y,x_{1}) \\
    C(x,x_{1}) C(y,x_{1})   & C(y,x_{1})^2
    \end{bmatrix}.
\end{align*}
Thus for a stationary covariance function $C$, we have the formula
\[
    C(x,x \mid x_{1}) = C(0) - \frac{C(x,x_{1})^2}{C(0)}.
\]

\begin{lemma}[Bounding the variance]
    ~\label{lemma:bounding_variance}
    Take a Matérn covariance Kriging process with measurements $x_{1}, \dots, x_{n} \in T$.
    Let $B \subseteq T$ be a set, such that for every $x \in B$ there exists a measurement point $x_{i}$ with $\lvert x-x_{i} \rvert \leq \varepsilon$.
    The variance of the process is bounded by  
    \[
    C(x,x \mid x_{1}, \dots x_{n}) \leq 2k \varepsilon^{\eta} 
    \] 
    with the constants $k, \eta$ from \cref{lemma:matern_hoelder} for all $x \in B$
\end{lemma}
\begin{proof}
To prove this bound we will combine the monotonicity of \cref{lemma:kriging_variance_monotone} with the bound of \cref{lemma:matern_hoelder}.
For any $x \in B$ pick the closest measurement $x_{i}$. Then
\begin{align*}
    C(x,x \mid x_{1}, \dots x_{n}) &\leq C(x,x \mid x_{i})\\
    &= \frac{C(0)^2-C(x,x_{i})^2}{C(0)}\\
    &= \frac{(C(0)-C(x,x_{i}))(C(0)+C(x,x_{i}))}{C(0)}\\
    &\leq 2 (C(0) - C(x,x_{i})).
\end{align*}
The last inequality holds because of the decreasing nature of the Matérn covariance $C(0)\geq C(x,x_{i})$.
We can now apply \cref{lemma:matern_hoelder} and have
\[
    C(x,x \mid x_{1}, \dots x_{n}) \leq 2k \lvert x-x_{i} \rvert^{\eta}
\]
with constants $k, \eta$ from \cref{lemma:matern_hoelder}.
Knowing that our measurement covers the space well enough, we can therefore uniformly bound the Kriging variance with
\[
    C(x,x \mid x_{1}, \dots x_{n}) \leq 2k \varepsilon^{\eta}. \qedhere
\]
\end{proof}


%%%%%%%%%%%%%%%%%%%%%%%%%%%%%%%%%%%%%%%%%%%%%%%%%%%%%%%%%%%%%%%%%%%%%%%%%%%%%%%%%%%%%%%%%%%%%%
%%%%%%%%%%%%%%%%%%%%%%%%%%%%%%%%%%%%%%%%%%%%%%%%%%%%%%%%%%%%%%%%%%%%%%%%%%%%%%%%%%%%%%%%%%%%%%

\section{Bounds on the chance of further improvement}



\begin{theorem}[Chance of further improvement with Dudley]
Let $ \{ X_{t} \}_{t \in T}$ be a Kriging process with measurements $x_{1}, \dots, x_{n}$  and Matérn covariance $C$ with parameters $\nu,m$.
Let $B \subseteq T$ be a set, such that for every $x \in B$ there exists a measurement point $x_{i}$ with $\lvert x-x_{i} \rvert \leq \varepsilon$ for some $\varepsilon$.
Define $\bar{\sigma}^2= \sup_{x \in B} C(x,x \mid x_{1}, \dots, x_{n})$, as well as the constant $A$ and $\alpha$ as in \cref{prop:covering_numbers} applied to $B \subseteq T$.
Then $\bar{\sigma}^2 \leq 2k \varepsilon^{\eta}$ and
the probability of finding a value greater than $h \geq 12 A \sqrt{\pi} \sqrt{\alpha}$ is bounded by
\[
    \Prob(\sup_{t \in B} X_{t} > h) \leq 2 \exp\left( \frac{(h-12 A \sqrt{\pi} \sqrt{\alpha})^2}{4\pi^2 k \varepsilon^{\eta}} \right).
\]
\end{theorem}

\begin{proof}
We can apply \cref{thm:dudley_int_ineq} to $B \subseteq T$ to get
\[
    \E \left[ \sup_{t \in B} \lvert X_{t} \rvert \right]
    \leq
    24\int_{0}^{\infty} \sqrt{\log (\mathcal{N}(B,d_{X},\varepsilon) )}\dx[\varepsilon].
\]
The covering numbers for \( \varepsilon \geq 0 \) are bounded due \cref{prop:covering_numbers} with $\mathcal{N}(B,d_{X},\varepsilon) \leq \left( \frac{A}{\varepsilon}  \right)^{\alpha} $.
For \( \varepsilon \geq A \) this is bounded by $ \left( \frac{A}{\varepsilon}  \right)^{\alpha} \leq 1$, which means the covering number must be $1$ and we can estimate
\[
    \E \left[ \sup_{t \in B} \lvert X_{t} \rvert \right]
    \leq
    24\int_{0}^{A} \sqrt{\log \left(\frac{A}{\varepsilon}\right)^{\alpha}}\dx[\varepsilon].
\]
%
To solve this integral, we first note that
\[
    f(\varepsilon) = \sqrt{\log (\frac{A}{\varepsilon})^{\alpha}}
\]
is injective with 
\[ f^{-1}(y) = A e^{-\frac{y^2}{\alpha}} \]
Then we apply Fubini's theorem in order to integrate over the $y$-axis
\begin{align*}
    \int_{0}^{A} \sqrt{\log (\frac{A}{\varepsilon})^{\alpha}}\dx[\varepsilon] &= \int_{0}^{\infty} \int_{0}^{A} \chi_{y \leq f(\varepsilon)} \dx[\varepsilon]\dx[y]\\
    &= \int_{0}^{\infty} \int_{0}^{A} \chi_{f^{-1}(y) \leq \varepsilon} \dx[\varepsilon]\dx[y] \\
    &= \int_{0}^{\infty} f^{-1}(y) \dx[y]\\
    &= \int_{0}^{\infty} A e^{-\frac{y^2}{\alpha}} \dx[y]
\end{align*}
where $\chi$ is the indicator function.
This integral is a scaled Gaussian integral that can be computed by polar coordinates, leading to
\[
    \int_{0}^{\infty} A e^{-\frac{y^2}{\alpha}} \dx[y] = A \frac{\sqrt{\pi} \sqrt{\alpha}}{2}.
\]
Now we can simplify the estimate to
\[
    \E \left[ \sup_{t \in B} \lvert X_{t} \rvert \right]
    \leq 12 A \sqrt{\pi} \sqrt{\alpha}.
\]
%
%
Next we will us use the concentration inequality from \cref{prop:concentration_ineq}.
The first consideration is the seperability of $\{ X_{t} \}_{t \in B}$. Here we can use the fact, that the seperability the process is equivalent to the seperability of $(B,d_{X})$. Which is given due to the seperability of $B \subseteq T$ with the Euclidean norm, as well as the Hölder continuity of the canonical distance
\(
    d_{X}(x,y) \leq  2k  \lvert x-y \rvert^{\eta}.
\)
The second consideration is the centeredness. If $\{ X_{t} \}_{t \in B}$ is not centered, we can subtract the Kriging mean function $\mu_{t}$ and work with $\{ X_{t}-\mu_{t} \}_{t \in T}$ instead.
Due to \cref{lemma:bounding_variance} the variance $\bar{\sigma}^2$ is finite and bounded by $\bar{\sigma}^2 \leq 2k \varepsilon^{\eta}$.
Therefore by the concentration inequality
\[
    \Prob\left( \left\lvert\sup_{t \in B} \lvert X_{t}\rvert - \E \left[\sup_{t \in B} \lvert X_{t}\rvert\right]  \right\rvert > u \right)
    \leq 2 e^{-(u^2/2\pi^2\bar{\sigma}^2)}.
\]
for $u > 0$.
Further
\[
    \Prob\left( \sup_{t \in B} X_{t} > u + \E\left[ \sup_{t \in B} \lvert X_{t} \rvert \right] \right) \leq \Prob \left(  \left| \sup_{t \in B} \lvert X_{t} \rvert - \E \left[ \sup_{t \in B} \lvert  X_{t} \rvert \right] \right|> u \right).
\]
We can substitute $h = u + \E \left[ \sup_{t \in B} \lvert  X_{t} \rvert \right] $.
Putting the two inequalites together
\[
    \Prob(\sup_{t \in B} X_{t} > h) \leq 2 \exp\left( \frac{(h-\E\left[ \sup_{t \in B} \lvert X_{t} \rvert \right] )^2}{2 \pi^2 \bar{\sigma}^2} \right) \leq 2 \exp\left( \frac{(h-12 A \sqrt{\pi} \sqrt{\alpha})^2}{2\pi^2 \bar{\sigma}^2} \right)
\]
holds for $h \geq 12 A \sqrt{\pi} \sqrt{\alpha}$.
Plugging in the estimate for $\bar{\sigma}^2$ proves the desired inequality.
\end{proof}




%%%%%%%%%%%%%%%%%%%%%%%%%%%%%%%%%%%%%%%%%%%%%%%%%%%%%%%%%%%%%%%%%%%%%%%%%%%%%%%%%%%%%%%%%%%%%%
%%%%%%%%%%%%%%%%%%%%%%%%%%%%%%%%%%%%%%%%%%%%%%%%%%%%%%%%%%%%%%%%%%%%%%%%%%%%%%%%%%%%%%%%%%%%%%
%
%
\begin{theorem}[Chance of further improvement with Talagrand]
    Let $ \{ X_{t} \}_{t \in T}$ be a Kriging process with measurements $x_{1}, \dots, x_{n}$  and Matérn covariance $C$ with parameters $\nu,m$.
    Let $B \subseteq T$ be a set, such that for every $x \in B$ there exists a measurement point $x_{i}$ with $\lvert x-x_{i} \rvert \leq \varepsilon$ for some $\varepsilon$.
    Define $\bar{\sigma}^2= \sup_{x \in B} C(x,x \mid x_{1}, \dots, x_{n})$, as well as the constant $A$ and $\alpha$ as in \cref{prop:covering_numbers}.
    Further choose \( \tilde{A} > \max \{ \sqrt{ 2k \varepsilon^{\eta}}, A \} \).
    Then the probability of finding a value greater than $u$ is bounded by
    \[
        \Prob \left( \sup_{t \in B} X_{t} \geq u \right)
        \leq  \left(\frac{K A u}{\sqrt{\alpha} \sigma^2} \right)^{\alpha} e^{\frac{-u^2}{2\sigma^2}}
    \]
    for $u \geq \bar{\sigma}(1+\sqrt{\alpha})$.
\end{theorem}
\begin{proof}
    We will check all the neccessary conditions needed to apply \cref{thm:talagrand}.
    By \cref{lemma:bounding_variance} the variance of the process is bounded with $\sigma^2 \leq 2k \delta^{\eta}$. By the choice of $\tilde{A}$ we therefore have $\tilde{A} > \sigma$.
    Further we have the bound on the covering numbers
    \[
        \mathcal{N}(B,d_{X},\varepsilon) \leq \left( \frac{A}{\varepsilon} \right)^{\alpha} \leq \left( \frac{\tilde{A}}{\varepsilon} \right)^{\alpha}
    \]
    from \cref{prop:covering_numbers}.
    We can now apply \cref{lemma:bounding_variance}, which gives us 
    \[
        \Prob \left( \sup_{t \in B} X_{t} \geq u \right)
        \leq  \left(\frac{K A u}{\sqrt{\alpha} \sigma^2} \right)^{\alpha} e^{\frac{-u^2}{2\sigma^2}},
    \]
    for $u \geq \bar{\sigma}(1+\sqrt{\alpha})$
\end{proof}

%%%%%%%%%%%%%%%%%%%%%%%%%%%%%%%%%%%%%%%%%%%%%%%%%%%%%%%%%%%%%%%%%%%%%%%%%%%%%%%%%%%%%%%%%%%%%%
%%%%%%%%%%%%%%%%%%%%%%%%%%%%%%%%%%%%%%%%%%%%%%%%%%%%%%%%%%%%%%%%%%%%%%%%%%%%%%%%%%%%%%%%%%%%%%

\section{Algorithm}
The following is an example of how this bound could be implemented in an algorithm.
We start like regular Bayesian optimization, by finding an optimal measurement point according to some acquisition function.
But in between taking measurements we use the bound on the chance of further improvement to check, if there are areas where further improvement is unlikely.
We then prune away these areas and only search in the remaining parts. 
If calculating these bounds is numerically cheaper than searching in the full area, then there is a performance gain in comparison to regular Bayesian optimization defined in \cref{alg:bayes_opt}.
%
In each iteration we perform a measurement and try to prune away parts of the search space $T$, where improvement is no longer likely. This can be done recursively for finer and finer subsets. The variables $d_{i}$ control the depth of search after each measurement point. The value describes how many times we cut $T$ in half when looking for areas to rule out. One could set $d_{i}=0$ periodically if taking multiple measurements in a row is desired. This depends on the cost of measurement in comparison to the cost of computing the bounds on the chance of improvement.
To illustrate the procedure, we look at an example.
Take $T=[0,1]^{d}$. We can cut into $[0,0.5] \times [0,1]^{d}$ and $ [0.5,1] \times [0,1]^{d-1}$ on the first level, then $[0,0.5] \times [0,0.5] \times [0,1]^{d-2} $ and so on until we have exhausted all the dimensions and arrive at blocks like form $[0,0.5]^{d}$, spread around $T$. Then we can start by cutting the first components in half again and repeat.
% We will also have to calculate or upper bound $\max_{x \in T_{k}} d(x, \{ x_{1}, \dots, x_{n} \})$, which is the biggest distance of any point to the next measurement.
%
%
\algnewcommand{\Inputs}[1]{%
  \State \textbf{Inputs:}
  \Statex \hspace*{\algorithmicindent}\parbox[t]{.8\linewidth}{\raggedright #1}
}
\algnewcommand{\Initialize}[1]{%
  \State \textbf{Initialize:}
  \Statex \hspace*{\algorithmicindent}\parbox[t]{.8\linewidth}{\raggedright #1}
}
\begin{algorithm}[H]
    \caption{Branching augmented global optimization}\label{alg:proposed_alg_2}
    \begin{algorithmic}[1]
        \Inputs{
            surrogate model $\{ G_{t} \}_{t \in T}$\\
            acquisition function $\alpha$\\ 
            probability threshold $\gamma$\\
            pruning search depths $d_{i}$\\
            a function $f$
            a set $T$
            a scheme for cutting $T$ into half spaces recursively
        }  
        \Initialize{
            $T_{pr} \gets \emptyset$\\
            $\mathcal{D}_{0} \gets \emptyset$\\
            $g^{*} \gets - \infty$
        }
        \For {$i=1,2,\dots$}
        \State select next point
        $
            x_{i} \gets \argmax_{x \in T \setminus T_{pr} } \alpha(x,\mathcal{D}_{i-1})
        $
        \State evaluate $f(x_{i})$
        \If{$f(x_{i})> g^{*}$}
        \State $g^{*} \gets f(x_{i})$
        \EndIf

        \State extend the data set $\mathcal{D}_{i} \gets \mathcal{D}_{i-1} \cup \{ (x_{i}, f(x_{i})) \}$
        \State update the surrogate model $\{ G_{t} \}_{t \in T}$
        \For {$j=0,\dots, d_{i}$}
        \For{all depth $j$ half spaces $T_{k}$ of $T$ with $T_{k} \cap T_{pr} = \emptyset$}
        \If{$\Prob\left( \sup_{t \in T_{k}} G_{t}> g^{*} \right) <  \frac{\operatorname{vol}(T_{k})}{\operatorname{vol}(T)} \gamma$}
        \State $T_{pr} \gets T_{pr} \cup T_{k}$
        \EndIf
        \EndFor
        \EndFor
        \If{$T_{pr} = T$}
        \State terminate algorithm early
        \EndIf
        \EndFor
    \end{algorithmic}
\end{algorithm} 


